\section{Charged particle Schrödinger equation}

In this problem, you will now investigate the quantum mechanics of a charged particle in the
electrostatic field you found in Problem~\ref{sec:pe}. In particular, you will determine the
ground state wavefunction using the Lanczos algorithm. In addition to the electrostatic
potential, the charged particle will not propagate outside the square region or into the
internal square. This means the wavefunction will vanish outside the square and on its
boundaries, and for all points in the internal square. We choose our units and the particle
mass so that the time-independent Schrödinger equation is
%
\begin{equation}\label{eq:sch}
    \bigl( -\nabla^2 + q \phi \bigr) \psi = E \psi,
\end{equation}
%
where \(\phi(x, y)\) is the electrostatic potential that you found in Problem~\ref{sec:pe}.
We let the charged particle have a charge of \(q = 0.001\) in dimensionless units. Since
this equation is real, all of the eigenvectors are real.

Implement the Lanczos algorithm to find the lowest eigenvalue and eigenvector for this
system. Your matrix times vector function here is very similar to the one you wrote for
Problem~\ref{sec:pe}, but now there is an additional term, \(q \phi \psi\).
Thinking of the left-hand side of Equation~\eqref{eq:sch}
as a matrix operation, the \(q \phi \psi\) term is an additional diagonal element of the
matrix. You need to multiply \(q \phi\) and \(\psi\) at the same point for all points.

You may find the eigenvalues and eigenvectors of the tridiagonal Lanczos matrix, \(\mathrm{T}_k\),
which is generated from \(k - 1\) steps of the Lanczos recursion. Denote the smallest
eigenvalue of \(\mathrm{T}_k\) by \(E_0\) and let \(\bm{w}\) be the associated eigenvector, 
where \(\bm{w}\) is a vector of length \(k\).

Since the Lanczos is unstable if the iteration continues for too long, you should run it for
a small number of steps (\(30\) is a good place to start), compute the approximate lowest
eigenvalue and eigenvector of \(\mathrm{T}_k\) and restart the Lanczos again. When you restart the
Lanczos, use the current best approximation to the desired eigenvector of the full system,
which is
%
\begin{equation}\label{eq:restart}
    v'_n = \sum_i v_{n,i} w_i, \quad n = 1, \ldots, N,
\end{equation}
%
where \(N\) is the size of the vectors for the full problem and \(v_{n, i}\) is the
\(n\)th element of the \(i\)th
Lanczos vector. Check the accuracy of your lowest eigenvalue by applying the Laplacian plus
potential energy to the eigenvector to see if the result is the eigenvector scaled by the
eigenvalue. You will have to restart the Lanczos \(30 - 40\) times to get single precision
accuracy in the lowest eigenvalue.

\Question{} Once you can reliably find the lowest eigenvalues and eigenvectors, find the
probability that the particle is in the region \(y < 0.5\). Remember, the probability is
determined by the sum over \(y < 0.5\) of the wave function's norm squared. Make a 3D plot
of the probability, i.e., \(\abs{\psi(x, y)}^2\).
\newline
